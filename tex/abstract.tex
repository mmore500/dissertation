Evolutionary transitions occur when previously-independent replicating entities unite to form more complex individuals.
For example, a fraternal transition involves lower-level entities that are kin joining together into a higher-level self-replicating entity (e.g., transitions to multicellularity or to eusocial colonies).
Such major transitions in individuality have profoundly shaped complexity, novelty, and adaptation over the course of natural history and regard for their causes and consequences drives many fundamental questions in biology.
Likewise, evolutionary transitions have been highlighted as key to artificial life research developing systems capable of true open-ended evolution.
As such, experiments with digital multicellularity promise to help realize computational systems with properties that more closely resemble those of biological systems, ultimately providing insights about the origins of comlpex life in the natural world.

This dissertation proposal is divided into two parts. 
In the first part, I discuss the methodological stack that I am developing to enable dynamic, large-scale digital multicellularity experiments and analyses that leverage high performance computing hardware.
This stack includes four important tools:
(1) MatchBin, a tag-matching system to enable dynamic interaction in simple event-driven code,
(2) SignalGP-lite, efficient virtual CPUs that use MatchBin for event-driven genetic programing,
(3) the Conduit framework to allow highly scalable best-effort communication in distributed software, and
(4) a proposed algorithm capable of partial phylogenetic reconstruction under distributed, lossy conditions.

In the second part of this document these tools come together in the form of DISHTINY a distributed digital evolution system capable of producing and tracking populations of organisms as they undergo major evolutionary transitions in individuality.
I discuss the research that I have conducted using DISHTINY and I propose my next steps in this line of research.
The existing projects explore the breadth of multicelluar organisms produced and a case study delving into the evolutionary history of a single experiment.
Finally, I propose a study testing the effect of the event-driven virtual CPU architecture on the evolution of complex multicelluar strategies.
