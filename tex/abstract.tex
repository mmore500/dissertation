Evolutionary transitions occur when previously-independent replicating entities unite to form more complex individuals.
For example, a fraternal transitions involves lower-level entities that are kin joining into a self-replicating entity (e.g., transitions to multicellularity or to eusocial colonies).
Such transitions have profoundly shaped complexity, novelty, and adaptation over the course of natural history.
This reason drives keen interest in major evolutionary transitions and multicellularity in artificial life, particularly with respect to the topic of open-ended evolution.
As such, experiments with digital multicellularity promise to help realize computational systems with properties that more closely resemble those of biological systems.

In this dissertation, I have developed a methodological stack that enables dynamic, large-scale digital multicellularity experiments that leverage parallel and distributed high performance computing hardware.
I present tag-matching results and software improvements for the event-driven genetic programming paradigm to enable a suitable evolutionary substrate for digital multicellularity.
Next, I introduce a best-effort communication approach designed to enable highly scalable artificial life simulations.
Then, I describe a decentralized scheme to select for digital multicellularity and report case studies of qualitative evolved life histories as well as quantifying evolution of complexity and adaptation within a case study.
Finally, I test whether better responsiveness and plasticity observed under event-driven genetic programming extends into the artificial life context by quantifying adaptiveness and richness of environmental and agent-agent interactions.
