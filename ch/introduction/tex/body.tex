\chapter{Introduction}
\label{ch:introduction}

\noindent
Portions of this chapter are adapted from ~\citep{moreno2019toward}, ~\citep{moreno2020practical}, and from \citep{moreno2020profiling}.

\section{Major Evolutionary Transitions and Open-Ended Evolution}

Emergence of new replicating entities from the union of simpler entities constitutes some of the most profound events in natural evolutionary history \citep{smith1997major}.
In an evolutionary transition of individuality, a new, more complex replicating entity is derived from the combination of cooperating replicating entities that have irrevocably entwined their long-term fates \citep{west2015major}.
Eusocial insect colonies and multicellular organisms exemplify this phenomenon \citep{smith1997major}.
Such transitions in individuality are essential to the evolution of the most complex forms of life.
As such, these transitions have been highlighted as key research targets with respect to the question of open-ended evolution \citep{ray1996evolving, banzhaf2016defining}.

In particular, this dissertation focuses on fraternal transitions in individuality --- events where closely-related kin come together or stay together to form a higher-level organism \citep{queller1997cooperators}.
Potential evolvability properties of fraternal collectives makes them an attractive evolutionary substrate.
Multicellular bodies configured through generative development (i.e., with indirect genetic representation) can promote scalable properties \citep{lipson2007principles} such as modularity, regularity, and hierarchy \citep{hornby2005measuring, clune2011performance}.
Developmental processes may also promote canalization \citep{stanley2003taxonomy}, for example through exploratory processes and compensatory adjustments \citep{gerhart2007theory}.

Scientific understanding of fraternal transitions in individuality benefits from experimental work probing the origins of multicellularity.
In the biological domain, Ratcliff et al. have demonstrated evolution of multicellularity in yeast, deriving fraternal clusters of cells that cling together in order to maximize their settling rate \citep{ratcliff2012experimental}.
The contributions of Goldsby and collaborators are particularly notable among computational artificial life work on the origins of multicellularity.

Goldsby’s work extends the Avida model system \citep{ofria2009artificial}, breaking the toroidal grid into isolated pockets where colonies are grown up from a single progenitor cell.
Direct selection for collective, colony-level characteristics drives evolution of cooperative cellular traits characteristic of a transition to colony-level individuality.
When a colony meets selection criteria, a propagule from that colony is inoculated into a freshly-cleared population slot.
Cells explicitly self-designate eligibility to parent a propagule.
This clear distinction between somatogenic and gametogenic modes of reproduction has proven particularly useful in experiments studying the origin of soma \citep{goldsby2014evolutionary} and multicellular entrenchment \citep{goldsby2020major}.
Other work by Goldsby et al. has investigated the evolution of division of labor \citep{goldsby2010evolution, goldsby2012task} and the evolution of morphological development \citep{goldsby2017increasing}.

\section{Digital Evolution Models}

Digital evolution techniques complement traditional wet-lab evolution experiments by enabling researchers to address questions that would be otherwise limited by:
\begin{itemize}
\item reproduction rate (which determines the number of generations that can be observed in a set amount of time),
\item incomplete observations (every event in a digital system can be tracked),
\item physically-impossible experimental manipulations (any event in a digital system can can be arbitrarily altered), or
\item resource- and labor-intensity (digital experiments can be automated).
\end{itemize}

Despite their versatility and rapid generational turnover, digital artificial life experiments generally operate at comparable or modest scales compared to laboratory biological evolution experiments.
Although digital evolution techniques can feasibly simulate populations numbering in the millions, such experiments require simple agents with limited interactions.
With more complex agents controlled by genetic programs, neural networks, or the like, feasible population sizes can dwindle down to thousands or even hundreds of agents.
When considering major transitions to multicellularity --- where individual organisms are composed of many agents --- population sizes may drop to tens of organisms, far below desirable for many evolution experiments.

\section{Putting Scale in Perspective}

One example of a digital evolution platform is Avida, a popular software system for evolutionary experiments with self-replicating computer programs.
In this system, a population of ten thousand digital organisms can undergo approximately twenty thousand generations (or about two hundred million individual replication cycles) per day \citep{ofria2009artificial}.
Each flask in the Lenski Long-Term Evolution Experiment hosts a similar number of replication cycles; with an effective population size of 30 million \textit{E. coli} that undergo a bit more than 6.6 doublings per day, the bacteria experience about 180 million replication events per day \citep{good2017dynamics}.
Likewise, in Ratcliff’s work studying the evolution of multicellularity in \textit{S. cerevisiae}, about six doublings per day occur among a population numbering on the order of a billion cells \citep{ratcliff2012experimental}.
These numbers translate to approximately six billion cellular replication cycles elapsed per day in this system.

Although artificial life practitioners traditionally describe instances of their simulations as ``worlds,'' with serial processing power their scale aligns (in naive terms) more along the lines of a single flask.
Of course, such a comparison neglects profound disparities between Avidians and bacteria or yeast in terms of complexity.
Natural organisms have vastly more information content in their genomes and their cellular state, as well as more (and more diverse) interactions with the environment and with other cells.
Recent work with SignalGP has sought to address some of these shortcomings by developing digital evolution substrates suited to dynamic environmental and agent-agent interactions \citep{lalejini2018evolving} that more effectively incorporate state information \citep{lalejini2021tag,lalejini2020case, moreno2019evaluating}.
However, more sophisticated and interactive evolving agents will necessarily consume more CPU time on a per-replication-cycle basis --- further shrinking the magnitude of experiments tractable with serial processing.

\section{Thesis Statement}

\noindent\fbox{%
\parbox{\textwidth}{%
Scalable digital evolution systems leveraging best-effort communication will enable us to study key phenomena associated with open-ended evolution: the origins of novel traits and behaviors, complex organisms and ecologies, and major evolutionary transitions in individuality.
}%
}

\section{A Path of Expanding Computational Scale}

While by no means certain, the idea that orders-of-magnitude increases in compute power will open up qualitatively different possibilities with respect to open-ended evolution is both promising and well founded.
Spectacular advances achieved with artificial neural networks over the last decade illuminate a possible path toward this outcome.
As with digital evolution, artificial neural networks (ANNs) were traditionally understood as a versatile, but auxiliary methodology --- both techniques have been described as ``the second best way to do almost anything'' \citep{miaoulis2008intelligent,eiben2015introduction}.
However, the utility and ubiquity of ANNs has since increased dramatically.
The development of AlexNet is widely considered pivotal to this transformation.
AlexNet united methodological innovations from the field (such as big datasets, dropout, and ReLU) with GPU computing that enabled training of orders-of-magnitude-larger networks.
In fact, some aspects of their deep learning architecture were expressly modified to accommodate multi-GPU training \citep{krizhevsky2012imagenet}.
By adapting existing methodology to exploit commercially available hardware, AlexNet spurred the greater availability of compute resources to the research domain and eventually the introduction of custom hardware to expressly support deep learning \citep{jouppi2017datacenter}.

Notably within the domain of artificial life, David Ackley has envisioned an ambitious design for modular distributed hardware at a theoretically unlimited scale \citep{ackley2011pursue}.
Progress toward realizing artificial life systems with such indefinite scalability seems likely to unfold as incremental achievements that spur additional interest and resources in a positive feedback loop with the development of methodology, software, and eventually specialized hardware to take advantage of those resources.
In addition to developing hardware-agnostic theory and methodology, we believe that pushing the envelope of open-ended evolution will analogously require designing systems that leverage existing commercially-available parallel and distributed compute resources at circumstantially-feasible scales.

\section{The Future is Parallel}

Throughout much of the 20th century, serial processing enjoyed regular advances in computational capacity due to quickening clock cycles, burgeoning RAM and caches, and increasingly clever packing together of instructions during execution.
Since, however, performance of serial processing has bumped up against apparent fundamental limits to the current technological foundations of computing \citep{sutter2005free}.
Instead, advances in 21st century computing power have arrived largely via multiprocessing \citep[p.~55]{hennessy2011computer} and specialized hardware acceleration (e.g., GPU, FPGA, etc.) \citep{che2008accelerating}.
Contemporary high-performance computing clusters link multiprocessors and accelerators with fast interconnects to enable coordinated work on a single problem \citep[p.~436]{hennessy2011computer}.
High-end clusters already make hundreds of thousands or millions of cores available.
More loosely-affiliated banks of servers can also muster significant computational power.
For example, Sentient Technologies notably employed a distributed network of over a million CPUs to run evolutionary algorithms \citep{miikkulainen2019evolving}.
The availability of orders-of-magnitude greater parallel computing resources in ten and twenty years’ time seems probable, whether through incremental advances with traditional silicon-based technology \citep{gropp2013programming,dongarra2014applied} or via emerging, unconventional technologies such as bio-computing \citep{benenson2009biocomputers} and molecular electronics \citep{xiang2016molecular}.
Such emerging technologies could greatly expand the collections of computing devices that are feasible, albeit at the potential cost of component speed \citep{bonnet2013amplifying, ellenbogen2000architectures} and perhaps also component reliability.
Making effective use of massively parallel processing power may require fundamental shifts in existing programming practices.

\section{Traditional Approaches to Digital Evolution at Scale Favor Isolation}

Digital evolution practitioners have a rich history of leveraging distributed hardware.
It is common practice to distribute multiple self-isolated instantiations of evolutionary runs across multiple hardware units.
In scientific contexts, this practice yields replicate datasets that provide statistical power to answer research questions \citep{dolson2017spatial}.
In applied contexts, this practice yields many converged populations that can be scavenged for the best solutions overall \citep{hornby2006automated}.
Another established practice is to use ``island models'' where individuals are transplanted between populations residing on different pieces of distributed hardware.
Koza and collaborators’ genetic programming work with a 1,000-CPU Beowulf cluster typifies this approach \citep{bennett1999building}.

In recent years, Sentient Technologies spearheaded evolutionary computation projects on an unprecedented computational scale, comprising over a million CPUs and capable of a peak performance of 9 petaflops \citep{miikkulainen2019evolving}.
According to its proponents, the scale and scalability of this ``DarkCycle'' system was a key aspect of its conceptualization \citep{gilbert2015artificial}.
Much of the assembled infrastructure was pieced together from heterogeneous providers and employed on a time-available basis \citep{blondeau2009distributed}.
Unlike typical island models where selection occurs entirely independently on each CPU, this scheme transferred evaluation criteria between computational instances in addition to individual genomes \citep{hodjat2013distributed}.
Sentient Technologies also notably exploited a large pool of hardware accelerators (e.g., 100 GPUs) in work evolving neural network architectures by performing each candidate architecture's costly model training and evaluation process \citep{miikkulainen2019evolving}.

Existing parallel and distributed digital evolution systems typically minimize interaction between simulation components on disjoint hardware.
Such independence facilitates simple and efficient implementation.
This approach typically involves independent evaluation of sub-populations (i.e., island models) or individuals (i.e., primary-subordinate or controller-responder parallelism \citep{cantu2001master}).
Cases where evaluation of a single individual are parallelized often involve data-parallel evaluation over a set of independent test cases, which are subsequently consolidated into a single fitness profile \citep{harding2007fast_springer, langdon2019continuous}.

However, several notable parallel and distributed digital evolution systems have incorporated rich interactions between parallelized simulation components.
Harding applied GPU acceleration to cellular automata models of artificial development systems, which involve intensive interaction between spatially-distributed instantiation of a genetic program \citep{harding2007fast_ieee}.
Work on Network Tierra by Tom Ray featured arbitrary communication between digital organisms residing on different machines \citep{ray1995proposal}
More recently, in a continuation of much earlier work, Christian Heinemann's ongoing ALIEN project has leveraged GPU acceleration for perform physics-based simulation of soft body agents within a 2D arena \citep{heinemann2008artificial}.

\section{Open-Ended Evolution at Scale Should Prioritize Interaction}

We argue that open-ended artificial life systems should prioritize dynamic interactions between simulation elements situated across physically distributed hardware components.

Unlike most existing applications of distributed computing in digital evolution, open-ended evolution research demands dynamic interactions among distributed simulation elements.
Many of the important natural phenomena, including ecologies, co-evolutionary dynamics, and social behavior, all arise from interactions among individuals.
Likewise, at the scale of an individual organism, developmental processes and emergent phenotypic functionality necessitate dynamic interactions.

A best-effort communication model could enable maximization of available bandwidth \citep{byna2010best} while avoiding scaling issues typically associated with communication-intensive distributed computing \citep{cardwell2019extended}.
Under such a model, processes compute simulation updates unimpeded and incorporate communication from collaborating processes as it happens to become available in real time.
As stochastic algorithms performing computational search with a broad set of acceptable outcomes, many digital evolution simulations are well suited to such a best-effort approach.

\section{Digital Multicellularity Suits Distributed Computing}

Multicellularity poses an attractive model to harness distributed computing power for digital evolution.
The basic notion is to achieve simulation dynamics that outstrip the capabilities of individual hardware components via an interacting network of discrete cellular components simple enough to reside on individual pieces of hardware.
Indeed, early thinking around composing digital organisms of differentiated components revolved around the possibility of multithreading and multiprocessing.
However, this work eschews a spatial model for cellular interaction in favor of a logical approach where ``cellular'' threads traversed logical space within a replicating program \citep{ofria1999evolution,ray2000evolution}.

Only later did Goldsby's multicellularity experiments introduce a spatial model for digital multicellularity, in which cells composing each digital ``multicell'' occupied tiles in a unique two-dimensional subgrid \citep{goldsby2014evolutionary}.
The clonal colony of cells constituting each multicell exists within an isolated spatial domain provisioned by the simulation.
Two distinct modes of reproduction occur in these experiments:
(1) cells replicate within a multicell and
(2) multicells reproduce by sending a single cell to found a new organism --- the target multicell is sterilized then re-innoculated with the cell supplied by the parent.
Although Goldsby did not pursue hardware acceleration of cell components within a multicell, such a spatial approach could facilitate parallelization.
Assuming local interactions, cells in a spatial model communicate directly with relatively few other simulation elements (i.e., their neighbors).
Such a limitation suits a distributed computing approach.

In fact, at truly vast scales where physical distance between hardware components limits viable communication, simulation topology that maps into three-dimensional space will become highly advantageous.
This argument is a foundational tenet of Ackley's ``indefinite'' scalability \citep{ackley2011pursue}.
David Ackley's recent work on emergent digital protocells exemplifies algorithm engineering grounded in spatial considerations with respect to potential underlying distributed physical hardware \citep{ackley2018digital,ackley2019building}.

The approach presented in this dissertation extends Goldsby's spatial model of digital multicellularity by developing mechanics to enable arbitrary interactions between multicells (e.g., competition, parental care for offspring, etc.) within a unified spatial realm.
(The DISHTINY model incorporates other notable changes, as well, such as an event-driven genetic programming substrate and directionally-symmetric agent evaluation.)

\section{Contributions}

Deepening our scientific understanding of major evolutionary transitions in individuality provides crucial insight into how the remarkable diversity and complexity of biological life came to be and may yield facilitate replication of lifelike capabilities \textit{in silico}.
Digital evolution enables unique experimental approaches to investigate evolutionary questions, but computational limitations restrict the scope of systems that can be modeled.
Such practicalities are particularly cumbersome to digital models of multicellularity.
This dissertation develops and tests approaches to improve scalability of artificial life simulations and applies them to construct a scalable simulation system for digital multicellularity.
We then use this system to study the relationships between major transitions, complexity, novelty, and adaptation.

Contributions of this dissertation are:
\begin{itemize}
\item implementing and evaluating techniques for general-purpose best-effort high-performance computing,
\item developing and implementing methodology for scalable simulations of evolving digital multicells that allows for arbitrary interactions between multicells in a unified spatial realm,
\item HSTRAT
\item de novo production of complex multicellular organisms without employing a segregating topology to force such a transition.
\item demonstrating metrics that can efficiently quantify complexity and adaptation in an system with implicit selection dynamics, and
\item characterizing the evolution of complexity, novelty, and adaptation of digital multicells in an open-ended system.
\end{itemize}

The work described here aims to spur reciprocal innovations:
\begin{itemize}
\item distributed computing will allow us to evolve complex multicellular digital organisms, and
\item the unique objectives and latitude of artificial life will foster novel distributed computing techniques.
\end{itemize}

\section{Outline}

The remainder of this dissertation is divided up as followed:

Part \ref{part:infrastructure} describes computational infrastructure that was developed to enable scalable digital multicellularity experiments.
\begin{itemize}
\item Chapter \ref{ch:conduit} presents the Conduit library for best-effort high-performance computing, experimentally demonstrating the scalability benefits of the best-effort approach, and
\item Chapter \ref{ch:distributed-phylogeny} proposes and tests the ``hereditary stratigraphy'' approach to record phylogenetic information in decentralized artificial life experiments.
\end{itemize}

Part \ref{part:experiments} reports experiments performed using the DISHTINY digital multicellularity framework.
\begin{itemize}
\item Chapter \ref{ch:case-studies} surveys multicellular life histories evolved within the framework, and
\item Chapter \ref{ch:measuring-cna} studies the coevolution of complexity, novelty, and adaptation in a case study lineage.
\end{itemize}

Finally, Chapter \ref{ch:conclusion} provides concluding remarks and describes directions in which this research should continue.
