\chapter{Methods to Enable Decentralized Phylogenetic Tracking in a Distributed Digital Evolution System}
\label{ch:distributed-phylogeny}

\noindent
Authors: Matthew Andres Moreno, Emily Dolson, and Charles Ofria \\
This chapter is adapted from ~\citep{moreno2022hereditary}, which underwent peer review and appeared in the proceedings of the 2022 Conference on Artificial Life (ALIFE 2022).

This chapter presents a novel algorithm (``hereditary stratigraphy'') to facilitate reconstruction-based phylogenetic studies in digital evolution systems.
This approach enables efficient, accurate phylogenetic reconstruction with tunable, explicit trade-offs between annotation memory footprint and reconstruction accuracy.
We can estimate, for example, MRCA generation of two genomes within 10\% relative error with 95\% confidence up to a depth of a trillion generations with genome annotations smaller than a kilobyte.
Simulated inference over known lineages recovers up to 85.70\% of the information contained in the original tree using 64-bit annotations.

\subimport*{submodules/hereditary-stratigraph-concept/tex/}{text/body.tex}
