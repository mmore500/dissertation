\begin{figure}
\begin{center}

\begin{subfigure}[b]{\textwidth}
\centering
\includegraphics[width=0.6\textwidth]{submodules/dishtiny/binder/bucket=prq49/a=all_stints_all_series_profiles+endeavor=16/teeplots/bucket=prq49+endeavor=16+hue=series+transform=filter-Stint-mod10+viz=lineplot+x=stint+y=num-less-fit-under-state-perturbation+ext=}%
\caption{
Lines track individual replicates
}
\label{fig:state-interface-complexity-vs-stint-lineplot}
\end{subfigure}

\begin{subfigure}[b]{\columnwidth}
\centering
\includegraphics[width=0.6\textwidth]{submodules/dishtiny/binder/bucket=prq49/a=all_stints_all_series_profiles+endeavor=16/teeplots/bucket=prq49+endeavor=16+transform=filter-Stint-mod10+viz=swarmplot-boxplot+x=stint+y=num-less-fit-under-state-perturbation+ext=}
\caption{
Boxplots with individual replicates overlaid as dots
}
\label{fig:state-interface-complexity-vs-stint-boxplot}
\end{subfigure}

\caption{
State interface complexity for 40 replicates over 100 three hour evolutionary stints.
State interface complexity sums the number of distinct input or output registers (also used to dispatch event-driven cues) that are associated with fitness decrease when individually swapped between cells or rotated between cardinals within a cell.
Reported values were measured from a representative strain harvested at the end of each stint.
}
\label{fig:state-interface-complexity-vs-stint}

\end{center}
\end{figure}
