\begin{figure}
\begin{center}

\begin{subfigure}[b]{\textwidth}
\centering
\includegraphics[width=0.6\textwidth]{submodules/dishtiny/binder/bucket=prq49/a=all_stints_all_series_profiles+endeavor=16/teeplots/bucket=prq49+endeavor=16+transform=groupby-Series-mean+viz=regplot+x=fitness-complexity+y=fraction-mutations-that-are-deleterious+ext=}%
\caption{
Robustness measured as fraction mutations that are deleterious
}
\label{fig:robustness-vs-fitness-complexity-fraction-mutations-that-are-deleterious}
\end{subfigure}

\begin{subfigure}[b]{\columnwidth}
\centering
\includegraphics[width=0.6\textwidth]{submodules/dishtiny/binder/bucket=prq49/a=all_stints_all_series_profiles+endeavor=16/teeplots/bucket=prq49+endeavor=16+transform=groupby-Series-mean+viz=regplot+x=fitness-complexity+y=mean-mutating-mutant-fitness-differential+ext=}
\caption{
Robustness measured as mean fitness differential between mutation-enabled and mutation-disabled strains.
}
\label{fig:robustness-vs-fitness-complexity-mean-mutating-mutant-fitness-differential}
\end{subfigure}

\caption{
Robustness and fitness complexity for strains sampled from 40 replicates across 100 three hour evolutionary stints at 10 stint intervals.
Fitness complexity counts the number of genome sites that individually contribute to fitness.
Individual observations are fully independent, computed as means over stints per replicate trial.
Shaded regions are 95\% confidence intervals.
}
\label{fig:robustness-vs-fitness-complexity}

\end{center}
\end{figure}
