\section{Diagnostic Problem Benchmarking}

In order to viably serve as a specialized alternative to the original SignalGP for certain artificial life applications, SignalGP-Lite must match SignalGP's performance on benchmarks measuring responsivity and plasticity (the ability of programs to adapt to changes in their environment).
To verify SignalGP-Lite's aptitude on these tests, we replicated two canonical SignalGP experiments, reported below \citep{lalejini2018evolving,lalejini_tag-based_2021}.

\subsection{Changing Environment Problem}

\begin{figure}[h]

\centering
\includegraphics[width=0.5\textwidth]{submodules/signalgp-lite/joss/figures/solution-time-swarmplot}

\caption{%
Number of generations elapsed before a perfect solution was observed on the Changing Environment problem.
All replicates found a perfect solution.
} \label{fig:tts-changing}

\end{figure}


The Changing Environment Problem dispatched $K =$ 2, 4, 8, or 16 mutually-exclusive environmental signals with randomly generated labels.
Programs were tasked to respond to each signal with a unique response instruction \citep{lalejini2018evolving}.

A total of 100 replicate populations of 100 individuals were evolved for up to 10,000 generations.
Elite selection was used to choose the best-fit individual; roulette selection was used for the other 99.
Figure \ref{fig:tts-changing} shows the number of generations elapsed before a full solution was found.
SignalGP-Lite evolved full solutions to each problem within 3,500 generations in all 100 tested replicates.

In the $K=16$ case, we achieved a superior 100\% signal reproduction rate compared to an average of 32\% on the original SignalGP  implementation  (\citep{lalejini2018evolving} Figure 2).
We suspect this improvement occurred due to differences in how mutation, tag matching, and program initialization were performed, rather than an intrinsic difference between the libraries.

\subsection{Contextual Signal Problem}

\begin{figure}[h]

\centering
\includegraphics[width=0.5\textwidth]{submodules/signalgp-lite/joss/figures/solution-time-contextual-signal}

\caption{%
Number of generations elapsed before a perfect solution was observed on the Contextual Signal problem.
} \label{fig:tts-context}

\end{figure}


The Contextual Signal Problem assesses the ability of evolving programs to maintain memory of previously encountered signals.
In previous work, this problem was used to demonstrate an important use case of regulation instructions.
To solve this problem, programs must remember an initial signal (i.e., its ``context'') in order to respond appropriately to a second signal \citep{lalejini_tag-based_2021}.

We assigned each possible unordered input signal pair a unique response to then be performed by the organism.
We tested with 16 input signal pairs and 4 output responses.
Table 2 in \citep{lalejini_tag-based_2021} enumerates these sequences and responses.

A total of 20 replicates were evolved for up to 10,000 generations using a 16-way lexicase selection scheme \citep{spector2012assessment}, with each of the input signal pairs serving as a test case.
To evaluate each test case, programs were sent the first signal of each test case and given 128 virtual CPU cycles to process it.
After this, the second signal was sent.
After another 128 virtual CPU cycles, their response was recorded.

In order to save resources and computing time, as soon as a replicate evolved a fully-correct solution, their evolution was halted.
We excluded random number generating operations from the instruction set to ensure that programs could not solve the Contextual Signal Problem by chance.
Figure \ref{fig:tts-context} shows the number of generations elapsed before a full solution was found.
SignalGP-Lite evolved full solutions in half as many generations compared to SignalGP when regulation was enabled.
Moreover---as opposed to SignalGP---no replicates failed to reach a full solution in 10,000 generations.
This may be due to the original SignalGP experiment purging running modules between signals, while our replication did not.

The primary object of interest this benchmark, however, is the extent to which each library's problem solving capability improves with regulation instructions enabled.
As expected, with regulation disabled the performance of both libraries is significantly degraded.
SignalGP-Lite's median (or mean, depending on what's reported in the other paper) time to solution discovery with regulation disabled was XX generations.
SignalGP's was YY generations.
Indeed, without regulation, both libraries achieved similarly slow generations to solution discovery.
\footnote{
We were surprised to see SignalGP-Lite evolve full solutions in the absence of regulation.
SignalGP reached solutions sans regulation through global memory operations that allow coordination between modules.
Although SignalGP-Lite's baseline instruction set does include global memory, manual inspection revealed that solutions appear to have arisen through intricately timed busy loops.
}

These results confirm that SignalGP-Lite's module regulation system functions comparably to SignalGP's in terms of facilitating context-dependent behavior.
