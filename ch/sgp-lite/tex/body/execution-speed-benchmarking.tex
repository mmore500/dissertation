\section{Execution Speed Benchmarking}

\begin{figure}[h]

\centering
\includegraphics[width=0.5\textwidth]{submodules/signalgp-lite/joss/figures/wall-time-speedup}

\caption{%
Benchmarking results of 20 replicates shown as a times-speedup of wall time.
``Benchmark'' refers to the set of instructions tested (see TODO section).
} \label{fig:bench-wall}

\end{figure}


We performed a set of microbenchmarks --- a type of synthetic benchmark that measures execution time of software subcomponents --- to quantify the effectiveness of SignalGP-Lite's optimizations in accelerating evaluation of event-driven genetic programs.

Hardware caching size profoundly affects memory access time, which is key to computational performance \citep{skadron1999branch}.
In order to determine the relative performance of SignalGP and SignalGP-Lite at different cache levels, we benchmarked over different orders of magnitude of memory load by varying the number of virtual CPUs (agent counts) between from 1 and 32768.

We performed five microbenchmark experiments, reported below, to isolate how specific aspects of the library design influenced performance.
Analysis below focuses on wall time speedup.
However, supplementary Figure \ref{fig:raw-timings} shows raw wall-clock timings for these experiments.
For each microbenchmark experiment, the Google Benchmark library used an instrumented burn-in process to dynamically determine the number of successive calls to the focal code snippet necessary to take a stable measurement of its run time.

\subsection{control}

The control involves importing the library to benchmark, initializing agents, and then measuring the execution time of an empty loop. This experiment verifies the validity of our benchmarking process.
The 1x wall speedup (Figure \ref{fig:bench-wall}) confirms that further results are not inadvertently skewed by our experimental apparatus.

\subsection{nop}

A program consisting of 100 \texttt{nop} instructions is randomly generated.
(Although some \texttt{nop} operation variants advance the random number generator engine in order to minimize unintended side effects when substituted for another operation that itself advances the random number generator, no \texttt{nop} instructions used in this experiment did so.)
This benchmarks instruction execution overhead directly, as it successive \texttt{nop}'ss are the only call measured inside the benchmarking loop.
With this approach, the relative performance impact of SignalGP-Lite's byte-code interpreter can be compared to SignalGP's lambda-based instructions.

We observe an $8\times$ to $30\times$ speedup under SignalGP-Lite (Figure \ref{fig:bench-wall}).
The greatest speedup occurred at a relatively light memory footprint of 1024 agents.

\subsection{arithmetic}

A program consisting of 100 randomly-chosen arithmetic instructions (\texttt{add}, \texttt{subtract}, \texttt{multiply}, and \texttt{divide}) is generated.
This measures the performance impact of SignalGP-Lite's fixed-length array registers compared to SignalGP's variable-length vector registers.
This compile-time optimization streamlines register access at the cost of the ability to change the number of registers on the fly.
Since our aim is to only measure the performance effect of this optimization, no \texttt{nop} instructions are present in the generated program.

Figure \ref{fig:bench-wall} shows that incorporating this trade-off increases speedup to 20x to 50x.
The greatest speedup increase occurred at a relatively light memory footprint of 1024 agents.

\subsection{complete}

The complete benchmark adds control flow instructions to the prior benchmarks' instruction set.
Bitwise and logical operators, comparison instructions, and random number generation operations, are also included.
From this complete instruction set, a 100-instruction program is randomly generated.

The main goal of this benchmark is to determine the performance impact of omitting a function stack and implementing inner loops and conditionals in terms of `jump` instructions instead of nested code blocks.

SignalGP-Lite's stripped-down control flow model increases speedup to $30\times$ to $55\times$ compared to vanilla SignalGP (Figure \ref{fig:bench-wall}).
The greatest speedup occurred at a light memory footprint of 32 agents.

\subsection{sans\_regulation}

Regulation operations allow SignalGP and SignalGP-Lite programs to adjust which program modules are expressed in response to environmental signals.
Since regulation can invalidate some of SignalGP and SignalGP-Lite's caching optimizations, we wanted to measure timings without regulation enabled.

This benchmark measures the complete instruction set with regulation-related instructions excluded.

As shown on \autoref{fig:bench-wall}, this yields a $35\times$ to $47\times$ speed-up with respect to SignalGP.
The greatest speedup occurred at a light memory footprint of 32 agents.
From this, we can conclude that SignalGP-Lite offers performance improvements even on simulations that do not depend on regulation.
