\chapter{Conclusion}
\label{ch:conclusion}

\noindent
Authors: Matthew Andres Moreno and Charles Ofria \\
Portions of this chapter are adapted from ~\citep{moreno2020practical}

The extent of complexity, novelty, and adaptation in the natural world inspires perennial scientific curiosity to understand aspects of its causes and engineering ambition to replicate aspects its properties.
In this dissertation, I have leveraged parallel and distributed high performance computing to perform digital evolution experiments to study complexity, novelty, and adaptation among emergent multicells.

\section{Contribution}

In Part \ref{part:infrastructure} of this dissertation, I present novel methodology to enable evolution of digital multicells within an environment dominated by biotic selection.
This methodology focuses particularly on computational scalability, yielding software for and analyses of best-effort communication that can generalize to other artificial life projects and beyond.

Chapter \ref{ch:tag-matching} characterizes variational, geometric, and evolutionary properties of several tag-matching schemes for event-driven genetic programming \citep{lalejini2018evolving}.
I found that provides a nonlinear ``streak''-based metric \citep{downing2015intelligence} provided intermediate geometric and variational properties that, in some situations, provided an evolutionary advantage.

Chapter \ref{ch:signalgp-lite} provides a computationally efficient implementation of event-driven genetic programming for artificial life contexts.
Use of a bytecode interpreter and data structures with better spatial locality enabled a ten times speedup in some situations.

Chapter \ref{ch:conduit} implements and tests best-effort communication approaches on commercially available high performance computing hardware.
We find that median performance of several quality-of-service metrics remains stable scaling up to 256 processes.
In separate experiments, we demonstrate how the best effort approach can provide better quality solutions to a graph coloring problem within a fixed time limit.

Chapter \ref{ch:distributed-phylogeny} proposes an approach to record phylogenetic information in decentralized artificial life experiments.

Part \ref{part:experiments} reports digital evolution experiments performed to study complexity, novelty, and adaptation within digital multicells.

Chapter \ref{ch:case-studies} describes four qualitative evolved multicellular life histories that evolved and reports phenomena characteristic of multicellularity including reproductive division of labor, resource sharing within kin groups, resource investment in offspring groups, asymmetrical behaviors mediated by messaging, morphological patterning, and adaptive apoptosis.

Chapter \ref{ch:measuring-cna} tracks the co-evolution of novelty, complexity, and adaptation in a case study lineage.
This case study lineage contained ten qualitatively distinct multicellular morphologies, several of which exhibit asymmetrical growth and distinct life stages.
Contextualize the evolutionary history of these morphologies with measurements of complexity and adaptation suggests a loose, sometimes divergent, relationship can exist among novelty, complexity, and adaptation.

Chapter \ref{ch:influencing-cna} proposes work to test the influence of event-driven genetic programming representation on the evolution of multicellular complexity and adaptation.


\section{Future Work}

Important work remains across the breadth of the methodological stack employed in this dissertation.

Further research should develop more efficient tag matching procedures for large sets of operands.
Current techniques are $O(n)$ with respect to the number of operands because queries must calculate match distance to every operand.
A compartmentalized approach may enable better performance by restricting potential matches to a subset of operands.
If designed properly, compartmentalized tag matching could also promote modularity by restricting which query-operand matches are possible under tag mutations.
Organizing tag-matching compartments chronologically and biasing against queries or operands mutating into older compartments might facilitate complexification by protecting ancient tag-matching circuits.

Decentralized methods for diversity maintenance also remain an open question.
Although diversity maintenance can readily be performed on a per-process basis using a finite resource model, a question remains how to generalize this approach to a distributed context.
(In future work, the current per-process approach may not be sufficient if smaller per-process cell counts or larger group size reduces the number of multicells occupying a single process too far.)

Sexual recombination plays a central role in natural history \citep{smith1997major} and genetic programming \citep{o2009riccardo}.
However, work has yet to be performed on sexual recombination with event-driven genetic programming encodings.
Segregation of linear genetic code into distinct, tagged modules suggests promising possibilities for semantic crossover.
This could enable digital evolution experiments probing the intersection between fraternal transitions in individuality and the evolution of sex.

Direct efforts to evolve emergent multicellular functionality should also be pursued.
Multicellular motility could be selected for by increasing resource collection rate based with distance the site where a group originated.
``Trading`` between groups or specialization between groups could be selected for by introducing discrete tokens with resource value that differs among cells or among groups (perhaps determined via a hash of cell or group ID and token ID).

Such efforts could extend evolving multicells to solve simple pattern detection or game playing tasks.
This would require careful thinking about how to ``wire'' input/output controls into multicell collectives and how to make problem instances available on demand to multicells in a distributed setting.
The ability to show multicells outperforming individual unicells or even collections of unicells on such problems would be an exciting result.
Past work exploring the introduction of neuron-like cell-cell interconnects into the DISHTINY model could serve as a stepping stone toward these objectives \citep{moreno2020practical}

We are excited to see how open-ended model systems built on commercially-available distributed computational substrates will address questions about how biological complexity relates to fitness, genetic drift over elapsed evolutionary time, mutational load, genetic recombination (sex and horizontal gene transfer), ecology, historical contingency, and key innovations.
